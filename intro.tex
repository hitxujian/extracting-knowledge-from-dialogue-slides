%!TEX root = Intro-NLP-seminar.tex
\section{Introduction}

\begin{frame}
\frametitle{sample dialogue}
\begin{itemize}
\item[] 
\begin{itemize}
\item[] (\OP{cuisine->food}: ?, area: ?, price: ?)
\end{itemize}
\item[] System: \textit{What type of restaurant are you looking for?}
\begin{itemize}
\item[] \texttt{request(\OP{!cuisine!})}
\end{itemize}
\item[] User: \textit{I am looking for a japanese restaurant in the city center.}
\begin{itemize}
\item[] \texttt{inform(\OP{!cuisine!}:japanese, area:center)}
\end{itemize}
\item[] System: \textit{What price range do you prefer?}
\begin{itemize}
\item[] \texttt{request(price)}
\end{itemize}
\item[] User: \textit{I want something cheap.}
\begin{itemize}
\item[] \texttt{inform(price:cheap)}
\end{itemize}
\end{itemize}
\end{frame}

\begin{frame}
\frametitle{sample dialogue}
\begin{itemize}
\item[] 
\begin{itemize}
\item[] (cuisine: ?, area: ?, price: ?)
\end{itemize}
\item[] System: \textit{What type of restaurant are you looking for?}
\item[] User: \textit{I am looking for a japanese restaurant in the city center.}
\begin{itemize}
\item[] (\textcolor{red}{cuisine: japanese}, \textcolor{red}{area: center}, price: ?)
\end{itemize}
\item[] System: \textit{What price range do you prefer?}
\item[] User: \textit{I want something cheap.}
\begin{itemize}
\item[] (cuisine: japanese, area: center, \textcolor{red}{price: cheap})
\end{itemize}
\end{itemize}
\end{frame}

\begin{frame}
\frametitle{Why track a dialogue}
\begin{itemize}[<+->]
\item Dialogue agent tracks the progress and decides what action to take.
\begin{itemize}
\item \OP{?Essential information?}
\end{itemize}
\item Need to to represent the state \OP{?somehow?}
\item Track \OP{?obtained?} information (history)
\item \alert{What does the user probably want?} \OP{nerozumim}
\end{itemize}
\end{frame}

\begin{frame}
\frametitle{What is the Dialogue state}

\begin{itemize}[<+->]
\item Triple \texttt{(method, slot, value)}
\item State space of possible situations
\begin{itemize}
\item Each situation described e.g. by Dialog Act Item
\item Probability distribution over situations
\item \OP{Tracking just slots and values}
\end{itemize}
\item Example using Dialog Act Items:
	\begin{itemize}
	\item (\textcolor{red}{cuisine}: \textcolor{blue}{japanese}, \textcolor{red}{area}: \textcolor{blue}{center}, \textcolor{red}{price}: \textcolor{blue}{moderate})
	\end{itemize}
\end{itemize}
\end{frame}

%\begin{frame}
%\frametitle{Possible approaches\cite{zilka2013}}
%\begin{itemize}
%\item \textbf{Discriminative belief update}
%\begin{itemize}
%\item Belief state is a product of marginal probabilities of all pairs \textit{slot:value}
%\end{itemize}
%\item \textbf{Belief update with generative model}
%\begin{itemize}
%\item Distribution factorized into transition and observation model
%\item Bayesian network of possible values
%\end{itemize}
%\item Both can accumulate information 
%\item Discriminative model outperforms the generative one, however it is quite naive.
%\end{itemize}
%\end{frame}

\begin{frame}
\frametitle{DSTC2 Dataset\cite{Henderson2014a}}
\begin{itemize}
\item Challenge organized by University of Cambridge, dataset publicly available.
\item Each dialogue composed of \textit{turns} - pair of user and system utterances.
\item Annotated dialogues - gold DAI after each turn.
\end{itemize}

\end{frame}

\begin{frame}
\frametitle{DSTC2 Dataset}
\begin{itemize}
\item Train set contains 1612, development 506 and test 1117 dialogues
\item Dialogues were obtained by user-computer interaction
\item Mix of Dialogue Managers \OP{?Systems?} and ASR engines was used.
\begin{itemize}
\item Different setting used for each set \alert{$\rightarrow$ increased difficulty}
\item Reshuffled data significantly easier
\end{itemize}
\end{itemize}
\end{frame}
